\documentclass{article}
\usepackage{graphicx}
\usepackage[top=0.9in, bottom=1in, left=1.5in, right=1.5in]{geometry}
\usepackage[utf8]{inputenc}
\usepackage[icelandic]{babel}
\usepackage[T1]{fontenc}
\usepackage[sc]{mathpazo}
\usepackage[parfill]{parskip}
\renewcommand{\baselinestretch}{1.2}
\usepackage{booktabs,tabularx}
\usepackage{multirow}
\usepackage{enumerate}
\usepackage{adjustbox}
\usepackage{multicol}
\usepackage{xcolor}
\usepackage{algpseudocode}
\usepackage{tikz}
\usepackage{nicefrac}
\usepackage{changepage}
\usepackage{hyperref}
\usetikzlibrary{arrows, positioning, calc, graphs}
\usepackage{amsmath, amsfonts, amssymb, amsthm}
\usepackage{graphicx}
\usepackage{tikz}
\usepackage{minted}
\usemintedstyle{manni}
\hypersetup{
    colorlinks=true,
    linkcolor=blue,
    filecolor=magenta,      
    urlcolor=cyan,
    pdftitle={Overleaf Example},
    pdfpagemode=FullScreen,
    }
\title{Vefforritun 2 Einstaklingsverkefni}
\author{Ragnar Björn Ingvarsson, rbi3}
\tikzset{->, >=stealth', shorten >=1pt, node distance=2cm,thick, main node/.style={circle,draw,minimum size=3em}}

\begin{document}
\renewcommand\thepage{}

	\maketitle

	\newpage
	\setcounter{page}{1}
	\renewcommand\thepage{\arabic{page}}

	\section{Inngangur}
	Þar sem ég hef verið núna að læra japönsku í rúm 2 ár, þá hef ég rekið mig á þann mikla vanda að horfa á þætti með japönskum texta. 
	Til að ná því þarf annaðhvort að fá áskrift að mörgum streymisíðum til að fá nógu gott úrval eða leita til sjóræningjanna. Ég hef alltaf valið seinni kostinn 
	en hef orðið þreyttur á ferlinu sem felst í því. Fyrst þarf að hlaða niður þáttum, svo leita í gegn um síðu með textum, hlaða niður textunum og færa þá á 
	þægilegan stað, og svo loks opna þáttinn og draga inn textann. Sem er náttúrulega alveg óásættanlegt að þurfa að hafa fleiri en einn glugga opinn á sama tíma. 
	
	Þess vegna ákvað ég að búa til þessa síðu sem gerir manni kleift að leita að þáttum og textum á sama stað, og helst að eins lítið basl og hægt sé eigi sér stað. 
	\section{Tækni, tól og útfærsla}
	Til að ná þessu markmiði ákvað ég að búa til framenda sem væri þægilegur notkunar og fallegur því það er nú þegar til síða sem þjónar næstum því sama tilgangi 
	en hún lítur hreinlega bara hræðilega út. Til þess notaði ég Next.js með React og Tailwind því ég hef mesta reynslu af því hingað til og vil helst einbeita 
	mér meira að því að fullkomna útfærsluna frekar en að læra á ný tól. Einnig notaði ég \texttt{motion} fyrir flóknari kvikun en það var lítið notað.

	Til að streyma þáttum þá nota ég \href{https://github.com/consumet/api.consumet.org}{consumet} sem leitar að seríum frá núverandi streymisíðum og tekur hls 
	hlekkinn frá þeim og gefur mér hann. Hlekkurinn er að sjálfsögðu verndaður af CORS svo ég nota einhvers konar CORS proxy til að komast hjá því. Fyrir texta 
	nota ég síðuna \href{https://kitsunekko.net/}{https://kitsunekko.net/} og næ einfaldlega bara í html-ið frá síðunni og vinn úr því vegna þess að þeir virðast 
	ekki gefa út neina betri leið.

	Ég útfærði semsagt bara framenda og ég hugsa að ég hafi einnig útfært flóknari framenda, og svo notfærði ég mér líka \href{https://www.figma.com/design/AH3i5YjtNi5afQ72QmelZ6/Untitled?node-id=0-1&t=3JF5reaVuTWhW4Yw-1}{Figma skjal}, sem er nú nokkuð einfalt en 
ég fylgdi því og notaði það mikið sem fyrirmynd.

	\section{Hvað gekk vel}

	Allur framendinn gekk rosalega vel, það var mjög fátt sem virkaði illa en aðallega var bara vesen vegna þess að ég byrjaði að hanna forritið með desktop í 
	huga sem reyndist vera erfitt með Tailwind þar sem það vill hanna mobile-first. Svo ég breytti hugmyndafræðinni minni til að koma til móts við það og eftir 
	þá breytingu gekk útlitshönnun rosalega vel fyrir sig.

	\section{Hvað gekk illa}

	Svolítið margt gekk illa, en fyrst var það að ná að geta streymt þætti sem var ekki í minni eigu. Það tók rosalegan tíma að finna út úr því og láta það allt 
	virka því consumet er líka ekki með mjög góða skjölun og svona. Svo, það sem gekk aðallega illa var að vinna á texta-browserinum. Sá component var ótrúlega 
	erfiður að vinna við því, eins og ég komst að mjög fljótt, óskipulögð gögn eru erfið að vinna með. Kitsunekko er semsagt samansafn af user-uploaded textum, 
	sem gerir það að verkum að það er ekkert samræmi í titlum, þáttanúmerum eða neitt. Þess vegna þarf ég að keyra titilinn til að leita að í gegn um database sem 
	skilar mismunandi leiðum til að skrifa titilinn og svo leita að því með fuzzy search og allskonar sem manni myndi kannski ekki detta í hug um leið að myndi 
	verða vandamál. En til að gera þetta ekki of langt eru þetta helstu tveir erfiðleikar mínir við verkefnið.

	\section{Hvað var áhugavert}

	Eitt sem mér fannst rosalega áhugavert var hversu mikið Figma skjal hjálpar við sköpun framenda. Ég hafði ekki gert mér grein fyrir hversu mikið getur 
	komið fyrir ef það er ekki til fyrirmynd til að hanna eftir og hvað það flýtir fyrir að sjá nákvæmlega hvað þarf að gera.

	\section{Lokaorð}

	Þetta var ótrúlega skemmtilegt verkefni og ég er mjög sáttur með útkomuna. Ein viðbæta sem ég mun svo bæta við er að ná að finna einhverja leið til að 
	sjálfkrafa endurtímasetja texta því eins og er eru þeir nærri allir á vitlausum tíma. En það er seinnitímavandamál og eins og þetta er þá svínvirkar það. (svo 
	lengi sem textarnir eru góðir).


	
\end{document}
